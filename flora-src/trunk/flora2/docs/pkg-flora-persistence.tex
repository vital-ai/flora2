\chapter[Persistent Modules]{Persistent Modules\\{\Large by Vishal Chowdhary}}



\newcommand{\psm}{\mbox{PM}\xspace}

This chapter describes a \FLSYSTEM package that enables persistent modules.  A
\emph{persistent module} (abbr., \psm) is like any other \FLSYSTEM module
except that it is associated with a database. Any insertion or deletion of
base facts and rules
in such a module results in a corresponding operation on the
associated database. This data persists across \FLSYSTEM sessions, so the
inserted data and rules
that were present in such a module are restored when the system restarts and
the module is re-attached to the database.

This package connects to databases via the ODBC interface, so the ODBC
drivers for the chosen DBMS (database management system) must be installed.
  
The \texttt{persistentmodules} package is experimental and has been tested
only with MySQL and PostgresSQL on Linux and Windows. It is possible that
it also works with Oracle, DB2, and SQL Server. 

\section{PM Interface}

A module becomes persistent by executing a statement that associates the
module with an ODBC data source described by a DSN (\emph{data source
  name}).
To start using the module persistence
feature, first load the following package into some module. For instance:
%% 
\begin{verbatim}
  ?- [persistentmodules>>pm].
\end{verbatim}
%% 
The following API is available. Note that \texttt{pm} is just a name we
chose for the sake of an example. If you load {\tt
  persistentmodules} into some other module, say {\tt foo}, then {\tt foo}
should be used instead of {\tt pm} in the examples below.    
\index{attach method}
\index{attachNew method}
%%
\begin{itemize}
\item {\tt ?- ?Module[attach(?DSN,?DB,?User,?Password)]@pm.}\\
  This action associates the data source described by an ODBC {\tt DSN}
  with the module.  

  Not all DBMSs support the operation of replacing the
  DSN's database at run time through ODBC (the USE statement).
  For instance, MS Access or PostgresSQL do not.
  In this case, \texttt{?DSN} must be bound to \texttt{connectDSN + dbDSN}
  (separate DSNs for connecting to DBMS and for the actual work database),
  as explained in Section~\ref{sec-dsn-more}.

  For other DBMS, such as MySQL, SQL Server, and Oracle, {\tt ?DSN} can be
  just an atom that specifies the work database. 

  The {\tt ?User} and {\tt ?Password} must be bound to the user name and
  the password to be used to connect to the database.

  The database specified by the DSN must already exist and must be created
  by a previous call to the method \texttt{attachNew} described below.
  Otherwise, the operation is aborted.
  The database used in the {\tt attach}
  statement must not be accessed directly---only through the persistent
  modules interface.  The above statement will create the necessary tables
  in the database, if they are not already present.

  Note that the same database can be associated with several different
  modules. The package will not mix up the facts that belong to different
  modules.
\item {\tt ?- ?Module[attachNew(?DSN,?DB,?User,?Password)]@pm.}\\
  Like {\tt attach}, but a new database is created as specified by {\tt
    ?DSN}.  If the same database already exists, an exception of the form
  {\tt \FLDBEXCEPTION(?ErrorMsg)} is thrown.  (In a program, include
  {\tt flora\_exceptions.flh} to define {\tt \FLDBEXCEPTION}; in the
  shell, use the symbol {\tt \fldbexception}.)  This method creates
  all the necessary tables, if they are not already present.

  As with the \texttt{attach} method, some DDBMS, like PostgresSQL, will
  need the DSN to be specified as \texttt{connectDSN + dbDSN}.
  
  As with the \texttt{attach} method, some DDBMS, like PostgresSQL, will
  need the DSN to be specified as 
  
  Note that this command works only with database systems that understand
  the SQL command {\tt CREATE DATABASE}. For instance, MS Access does not
  support this command and will cause an error.
\item {\tt ?- ?Module[detach]@pm.}\\
  Detaches the module from its database. The module is no longer persistent
  in the sense that subsequent changes are not reflected in any database.
  However, the earlier data is not lost. It stays in the database and the
  module can be reattached to that database.
\item {\tt ?- ?Module[loadDB]@pm.}\\
  On re-associating a module with a database (i.e., when {\tt
    ?Module[attach(?DSN, ?DB,?User,?Password)]@pm} is called in a new
  \FLSYSTEM session), database facts previously associated with the module are
  loaded back into it (but rules are loaded).  However, since the database may be large, \FLSYSTEM
  does not preload it into the main memory. Instead, facts are loaded
  on-demand.  If it is desired to have all these facts in main
  memory at once, the user can execute the above command. If no previous
  association between the module and a database is found, an exception is
  thrown.
\item \texttt{?- set\_field\_type(?Type)@pm.} \\  
  By default, \FLSYSTEM creates tables with the VARCHAR field type because
  this is the only type that is accepted by all major database systems.
  However, ideally, the CLOB (character large object) type should be used
  because VARCHAR fields are limited to 4000-7000 characters, which is
  usually inadequate for most needs.
  Unfortunately, the different database systems differ in how they support
  CLOBs, so the above call is provided to let the user specify the field types
  that would be acceptable to the system(s) at hand. The call should be
  made \textbf{before} the method \texttt{attachNew} is used. Examples:
  %% 
\begin{verbatim}
     ?- set_field_type('TEXT DEFAULT NULL')@pm.   // MySQL, PostgreSQL
     ?- set_field_type('CLOB DEFAULT NULL')@pm.   // Oracle, DB2
\end{verbatim}
  %% 
\end{itemize}



Once a database is associated with the module, querying and insertion of
the data into the module is done as in the case of regular (transient)
modules.  Therefore \psm's provide a transparent and natural access to the
database and every query or update may, in principle, involve a database
operation.  For example, a query like {\tt ?- ?D[dept -> ped]@StonyBrook.}
may invoke the SQL {\tt SELECT} operation if module {\tt StonyBrook} is
associated with a database.  Similarly {\tt  insert\{a[b \fd
  c]@stonyBrook\}} and {\tt delete\{a[e \fd f]@stonyBrook\}}  will invoke
SQL {\tt INSERT} and {\tt DELETE} commands, respectively. Thus, \psm's provide
a high-level abstraction over the external database.

Note that if {\tt ?Module[loadDB]@pm} has been previously executed,
queries to a persistent module will \emph{not} access the database since
\FLSYSTEM will use its in-memory cache instead. However, insertion and
deletion of facts in such a module will still cause database operations.

\section{More on DSNs}\label{sec-dsn-more}

ODBC drivers to DBMS's can be divided into two categories:
%% 
\begin{enumerate}
\item  those that support the USE statement that can change database
  connection at run time (MySQL, Oracle, others); and
\item those that do not (PostgreSQL, MS Access).
\end{enumerate}
%% 
For the first category, a DSN can be as simple as
%% 
\begin{verbatim}
  [mydb]
  Driver   = MySQL
  Server   = localhost
  User     = xsb
\end{verbatim}
%% 
Note that the actual database 
is not specified, as the package sets it at run-time.
The above DSN is for MySQL;
other DBMS in the same category may require additional fields in a DSN.

Here is how the attachment commands look in this case:
%% 
\begin{verbatim}
  ?- foo[attachNew(mydb, foo, xsb, 'xyz%56G9U')]@pm.
  ?- foo[attach(mydb, foo, xsb, 'xyz%56G9U')]@pm.
\end{verbatim}
%% 

For the second category of DBMS, one needs
to define two DSNs: the first for connecting to the DBMS and creating the
work database, and the second for working with that database.
This is how they might look for PostgreSQL:
%% 
\begin{verbatim}
  [postw]
  # This is the DSN for connecting to PostgreSQL
  Driver   = PostgreSQL_Unicode
  User     = postgres
  Database = template1
  # For Postgres must be Servername, not Server!!!
  Servername = localhost

  [postw2]
  # This is the DSN for actually working with the database xsb2 to which a
  # PM is attached
  Driver   = PostgreSQL_Unicode
  User     = postgres
  Database = xsb2
  # For Postgres must be Servername, not Server!!!
  Servername = localhost
\end{verbatim}
%% 
Here the \texttt{Database} field is mandatory. The database in the first
DSN,
\texttt{template1}, is an existing DBMS, which we used for the first
connection in which the package will create the work database \texttt{xsb2}.
The second DSN uses the work database \texttt{xsb2}. We could not use 
that DSN for connection because in \texttt{attachNew} the database
\texttt{xsb2} did not exist yet. (MySQL does not have that problem, as it
allows one to omit the \texttt{Database} field.)

This is how the attachment commands look like in the second case:
%% 
\begin{verbatim}
  ?- foo[attachNew(postw+postw2, xsb2, postgres, 'xyz%56G9U')]@pm.
  ?- foo[attach(postw+postw2, xsb2, postgres, 'xyz%56G9U')]@pm.
\end{verbatim}
%% 


\section{Examples}

The following example puts everything together.
This example works with MySQL-like databases.
For Postgres, use the attachment commands shown above.

%%
\begin{alltt}
// Create new modules mod, db_mod1, db_mod2.
\prompt newmodule\{mod\}, newmodule\{db_mod1\}, newmodule\{db_mod2\}.
\prompt [persistentmodules>>pm].

// pre-insert preliminary data into all three modules.
// Once db_mod1/db_mod2 are attached to a DB, the pre-inserted data will be
// added to the database (but not pre-inserted rules)
\prompt insert\{q(a)@mod,q(b)@mod,p(a,a)@mod\}.
\prompt insert\{p(a,a)@db_mod1, p(a,b)@db_mod1\}.
\prompt insert\{q(a)@db_mod2,q(b)@db_mod2,q(c)@db_mod2\}.

//  Associate modules db_mod1, db_mod2 with an EXISTING database db
//  To attach db_mod1/db_mod2 to a NEW database, use attachNew(...)
//  The data source is described by the DSN mydb.
\prompt db_mod1[attach(mydb,db,myuser,mypwd)]@pm.
\prompt db_mod2[attach(mydb,db,myuser,mypwd)]@pm.

// insert more data into db_mod2 and mod.
\prompt insert\{a(p(a,b,c),d)@db_mod2\}.
\prompt insert\{q(a)@mod,q(b)@mod,p(a,a)@mod\}.

// Both pre-inserted and the latest data will now be in the database

// shut down the engine
\prompt \bs{}halt.
\end{alltt}
%%

\noindent
Restart the \FLSYSTEM system.

%%
\begin{alltt}
// Create the same modules again
\prompt newmodule\{mod\}, newmodule\{db_mod1\}, newmodule\{db_mod2\}.

// try to query the data in any of these modules.
\prompt q(?X)@mod.
No.

\prompt p(?X,?Y)@db_mod1.
No.

//  Attach the earlier database to db_mod1.
\prompt [persistentmodules>>pm].
\prompt db_mod1[attach(mydb,db,user,pwd)]@pm.

// try querying again...

// Module mod is still not associated with any database and nothing was
// inserted there even transiently, we have:
\prompt q(?X)@mod.
No.

// But the following query retrieves data from the database associated
// with db_mod1.
\prompt p(?X,?Y)@db_mod1.
?X = a,
?Y = a.

?X = a,
?Y = b.

Yes.

// Since db_mod2 was not re-attached to its database,
// it still has no data, and the query fails.
\prompt q(?X)@db_mod2.

No.
\end{alltt}
%%

\section{Unsupported Features}\label{sec-persist-unsupport}

Insertion and deletion of rules into persistent modules is supported, 
but insertion/deletion of latent queries into such modules 
might not be fully supported
(this should work but has not been tested extensively).

Unlike facts inserted into modules,
if a module had rules inserted into it before the module is attached to a
database,
those rules will not be automatically inserted into that database and thus
will not persist from session to session.

Also, static rules---those appearing in loaded files (as opposed to the
added files)---cannot be made persistent.

Finally, disabled rules that are saved in a database get re-enabled
after they are loaded back into main memory. That is, rule disablement is
not persistent.

%%% Local Variables: 
%%% mode: latex
%%% TeX-master: "flora-packages"
%%% End: 
